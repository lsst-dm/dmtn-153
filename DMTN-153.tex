\documentclass[DM,authoryear,toc]{lsstdoc}
% lsstdoc documentation: https://lsst-texmf.lsst.io/lsstdoc.html
\input{meta}

% Package imports go here.

% Local commands go here.

%If you want glossaries
%\input{aglossary.tex}
%\makeglossaries

\title{Schema Management in DM}

% Optional subtitle
% \setDocSubtitle{A subtitle}

\author{%
William O'Mullane,
Colin Slater
}

\setDocRef{DMTN-153}
\setDocUpstreamLocation{\url{https://github.com/lsst-dm/dmtn-153}}

\date{\vcsDate}

% Optional: name of the document's curator
% \setDocCurator{The Curator of this Document}

\setDocAbstract{%
This note attempts to describe what we mean be schema management, what we would like  and how it is currently implemented within DM. 
}

% Change history defined here.
% Order: oldest first.
% Fields: VERSION, DATE, DESCRIPTION, OWNER NAME.
% See LPM-51 for version number policy.
\setDocChangeRecord{%
  \addtohist{1}{YYYY-MM-DD}{Unreleased.}{William O'Mullane}
}


\begin{document}

% Create the title page.
\maketitle
% Frequently for a technote we do not want a title page  uncomment this to remove the title page and changelog.
% use \mkshorttitle to remove the extra pages


\section{Types of Schemas}

\textbf{Abstract schema} --- DPDD (LSE-163). This describes the scientific content of the tables at a
level of detail that the future-user can understand what types of measurements will be produced,
without necessarily specifying the exact format of the resulting data. The line between these two
levels of detail is inherently blurry; users look to the DPDD to evaluate if project plans are
sufficient for their science, but the level of detail required for that evaluation sometimes
requires describing implementation choices that may change.

The abstract schema cannot by itself be realized in a concrete table; it lacks the full complement
of columns, data type information, and unique column names, and thus requires further elaboration.

\textbf{``Specification'' schemas} --- LDM-153. This is a schema that \textit{can} be physically
realized, and it is designed to fulfill the needs of the DPDD's abstract schema. This schema is
necessary both for sizing purposes, and as a ``goal'' that the pipelines teams can work to as they
build and evolve the pipeline output files. The result of the construction project should be for the
pipelines to produce data that fully realize this schema.

Because changes to the ``specification'' schema potentially have impacts on multiple areas of DM (e.g.
storage costs or science impacts), it is change-controlled at the DM-CCB level.


\textbf{``Concrete'' schemas} --- e.g. HSC reprocessing schemas. These schemas \textit{are}
physically realized; they are meant to describe data products that currently exist. These schemas
must accurately reflect those data products, regardless of what is specified in the DPDD or LDM-153.
They are not subject to change control since there is no project management impact that can be
caused by any changes. Inaccuracies may cause different dependent services to break, but this is
generally comparable in consequence to any other code breakage.

\section{Management of Schemas}


The DPDD is a project-level change-controlled document; procedures for modifying it are outside the
scope of this document.

The ``specification'' schema in LDM-153 should be updated whenever necessary to reflect any changes
in DM's planned data products at the end of construction. Evolution of the LDM-153 schema is
expected during construction as the pipelines evolve and the measurement outputs are better
understood.

LDM-153 is change-controlled by the DM-CCB, and its contents are generated from the baselineSchema.yaml file
in the \texttt{sdm\_schemas} repository. Procedurally, \texttt{baselineSchema.yaml} should be
treated like any other LaTeX input for a change controlled document: changes to it may be merged to
master via a normal ticket, but are not ``official'' until an approved RFC releases a new version of
the LDM document.

The \texttt{hsc.yaml} schema in \texttt{sdm\_schemas} is a ``concrete'' schema that is not subject
to change control, but the \texttt{ci\_hsc} integration tests verify that the outputs from that
pipeline execution comply with the schema specified in \texttt{hsc.yaml}. This ensures that an
up-to-date schema is always available, so that steps like loading an HSC reprocessing run into qserv
do not require fixing up all the changes to the schema since the previous ingest.

All other schemas in the \texttt{sdm\_schemas} repository are ``concrete'' schemas reflecting
specific sets of data products; these may be edited as necessary by a normal ticket workflow.

\textbf{TODO:} we only have one hsc.yaml, should we be creating more reprocessing-specific copies?
I.e. one for each RC2 reprocessing and saving them in separate files in sdm\_schemas.

\section{Schema File Format}

The ``concrete'' schema information needs to be available to a variety of different tools, each with
slightly different needs. Because of this, it was not sufficient to adopt a format like SQL
\texttt{CREATE TABLE} statements that were only suited to one particular use, and difficult to parse
for all other uses. Instead, the schemas in the \texttt{sdm\_schemas} repository are in a yaml
format defined by the \texttt{Felis} tool. The ease of parsing yaml makes it possible for many
different tools to all share the same source of schema information, minimizing intermediate stages.

The current uses of the Felis-defined yaml files are:
\begin{enumerate}
\item Qserv ingest --- The Felis files are used as inputs to the ingest process.

\item TAP\_SCHEMA creation --- The Felis tool itself is designed to generate SQL statements that
    populate the schema database used by the TAP standard.

\item LDM-153 generation --- The tables in the document generated from the Yaml files.

\item Pipeline data product verification --- Continuous integration tests verify that the pipeline
    outputs comply with the physical schema in \texttt{hsc.yaml}.
\end{enumerate}

Most of these uses depend on the YAML schema files without relying on the Felis tool itself. This is
generally a consequence of the YAML format being easy to parse by other tools.


\section{Science Pipelines}

Data products that are generated by the science pipelines must conform to a physical Felis-defined
schema in order to be loaded into databases. The pipelines generate a variety of intermediate
catalog products, often in afwTable FITS files, which must be transformed into the user-facing
tables by the pipelines team. This transformation is more than naming and units: e.g. fluxes may
need to have calibrations applied, or uncertainties in pixel units may be transformed to angular
units by using WCS information.

This transformation process is not change controlled; the pipelines team has full control over how
the user-facing tables are generated. It is only the definition of the user-facing tables that is
change-controlled.

The pipelines code implements the various column transformations via a series of ``functors'', each
dedicated to a particular type of transformation, and the definition of which functors are applied
to which columns is defined by a YAML configuration file. This YAML file has an entirely distinct
format from the Felis-defined schemas, and its purpose is distinct. The pipelines code generates
data products that comply with a particular Felis schema, but there is no automatic linkage between
the two. Simple verification can be performed on the pipelines' YAML file to ensure that it
generates columns with the correct names, but any changes to the output data products require both
the Felis YAML and the pipelines' YAML to have corresponding updates applied.



\appendix
% Include all the relevant bib files.
% https://lsst-texmf.lsst.io/lsstdoc.html#bibliographies
\section{References} \label{sec:bib}
\renewcommand{\refname}{} % Suppress default Bibliography section
\bibliography{local,lsst,lsst-dm,refs_ads,refs,books}

% Make sure lsst-texmf/bin/generateAcronyms.py is in your path
\section{Acronyms} \label{sec:acronyms}
\addtocounter{table}{-1}
\begin{longtable}{p{0.145\textwidth}p{0.8\textwidth}}\hline
\textbf{Acronym} & \textbf{Description}  \\\hline

ADQL & Astronomical Data Query Language \\\hline
CI & Continuous Integration \\\hline
CSV & Comma Separated Values \\\hline
DM & Data Management \\\hline
DMTN & DM Technical Note \\\hline
DPDD & Data Product Definition Document \\\hline
IVOA & International Virtual-Observatory Alliance \\\hline
LSP & LSST Science Platform \\\hline
LSST & Legacy Survey of Space and Time (formerly Large Synoptic Survey Telescope) \\\hline
PPDB & Prompt Products DataBase \\\hline
SEM & Systems Engineering Manager \\\hline
SQL & Structured Query Language \\\hline
TAP & Table Access Protocol \\\hline
VO & Virtual Observatory \\\hline
YAML & Yet Another Markup Language \\\hline
\end{longtable}

% If you want glossary uncomment below -- comment out the two lines above
%\printglossaries





\end{document}
