\documentclass[DM,authoryear,toc]{lsstdoc}
% lsstdoc documentation: https://lsst-texmf.lsst.io/lsstdoc.html
\input{meta}

% Package imports go here.

% Local commands go here.

%If you want glossaries
%\input{aglossary.tex}
%\makeglossaries

\title{Schema Management in DM}

% Optional subtitle
% \setDocSubtitle{A subtitle}

\author{%
William O'Mullane
}

\setDocRef{DMTN-153}
\setDocUpstreamLocation{\url{https://github.com/lsst-dm/dmtn-153}}

\date{\vcsDate}

% Optional: name of the document's curator
% \setDocCurator{The Curator of this Document}

\setDocAbstract{%
This note attempts to describe what we mean be schema management, what we would like  and how it is currently implemented within DM. 
}

% Change history defined here.
% Order: oldest first.
% Fields: VERSION, DATE, DESCRIPTION, OWNER NAME.
% See LPM-51 for version number policy.
\setDocChangeRecord{%
  \addtohist{1}{YYYY-MM-DD}{Unreleased.}{William O'Mullane}
}


\begin{document}

% Create the title page.
\maketitle
% Frequently for a technote we do not want a title page  uncomment this to remove the title page and changelog.
% use \mkshorttitle to remove the extra pages


\section{Types of Schemas}

\textbf{Abstract schema} --- DPDD (LSE-163). This describes the scientific content of the tables at a
level of detail that the future-user can understand what types of measurements will be produced,
without necessarily specifying the exact format of the resulting data. The line between these two
levels of detail is inherently blurry; users look to the DPDD to evaluate if project plans are
sufficient for their science, but the level of detail required for that evaluation sometimes
requires describing implementation choices that may change.

The abstract schema cannot by itself be realized in a concrete table; it lacks the full complement
of columns, data type information, and unique column names, and thus requires further elaboration.

\textbf{``Specification'' schemas} --- LDM-153. This is a schema that \textit{can} be physically
realized, and it is designed to fulfill the needs of the DPDD's abstract schema. This schema is
necessary both for sizing purposes, and as a ``goal'' that the pipelines teams can work to as they
build and evolve the pipeline output files. The result of the construction project should be for the
pipelines to produce data that fully realize this schema.

Because changes to the ``specification'' schema potentially have impacts on multiple areas of DM (e.g.
storage costs or science impacts), it is change-controlled at the DM-CCB level.


\textbf{``Concrete'' schemas} --- e.g. HSC reprocessing schemas. These schemas \textit{are}
physically realized; they are meant to describe data products that currently exist. These schemas
must accurately reflect those data products, regardless of what is specified in the DPDD or LDM-153.
They are not subject to change control since there is no project management impact that can be
caused by any changes. Inaccuracies may cause different dependent services to break, but this is
generally comparable in consequence to any other code breakage.


\section{Management of Schemas}


The DPDD is a project-level change-controlled document; procedures for modifying it are outside the
scope of this document.

The ``specification'' schema in LDM-153 should be updated whenever necessary to reflect any changes
in DM's planned data products at the end of construction. Evolution of the LDM-153 schema is
expected during construction as the pipelines evolve and the measurement outputs are better
understood.

LDM-153 is change-controlled by the DM-CCB, and its contents are generated from the baselineSchema.yaml file
in the \texttt{sdm\_schemas} repository. Procedurally, \texttt{baselineSchema.yaml} should be
treated like any other LaTeX input for a change controlled document: changes to it may be merged to
master via a normal ticket, but are not ``official'' until an approved RFC releases a new version of
the LDM document.

The \texttt{hsc.yaml} schema in \texttt{sdm\_schemas} is a ``concrete'' schema that is not subject
to change control, but the \texttt{ci\_hsc} integration tests verify that the outputs from that
pipeline execution comply with the schema specified in \texttt{hsc.yaml}. This ensures that an
up-to-date schema is always available, so that steps like loading an HSC reprocessing run into qserv
do not require fixing up all the changes to the schema since the previous ingest.

All other schemas in the \texttt{sdm\_schemas} repository are ``concrete'' schemas reflecting
specific sets of data products; these may be edited as necessary by a normal ticket workflow.

\textbf{TODO:} we only have one hsc.yaml, should we be creating more reprocessing-specific copies?
I.e. one for each RC2 reprocessing and saving them in separate files in sdm\_schemas.

\textbf{TODO:} should mention how felis is involved, but it's mostly as a definition for the schema
yaml format and doesn't need to be the main focus.

\textbf{TODO:} maybe describe how the pipeline's ``SDM standardization'' yaml file relates to these
schemas? Some version of the Princeton whiteboard diagram?


\appendix
% Include all the relevant bib files.
% https://lsst-texmf.lsst.io/lsstdoc.html#bibliographies
\section{References} \label{sec:bib}
\renewcommand{\refname}{} % Suppress default Bibliography section
\bibliography{local,lsst,lsst-dm,refs_ads,refs,books}

% Make sure lsst-texmf/bin/generateAcronyms.py is in your path
\section{Acronyms} \label{sec:acronyms}
\addtocounter{table}{-1}
\begin{longtable}{p{0.145\textwidth}p{0.8\textwidth}}\hline
\textbf{Acronym} & \textbf{Description}  \\\hline

ADQL & Astronomical Data Query Language \\\hline
CI & Continuous Integration \\\hline
CSV & Comma Separated Values \\\hline
DM & Data Management \\\hline
DMTN & DM Technical Note \\\hline
DPDD & Data Product Definition Document \\\hline
IVOA & International Virtual-Observatory Alliance \\\hline
LSP & LSST Science Platform \\\hline
LSST & Legacy Survey of Space and Time (formerly Large Synoptic Survey Telescope) \\\hline
PPDB & Prompt Products DataBase \\\hline
SEM & Systems Engineering Manager \\\hline
SQL & Structured Query Language \\\hline
TAP & Table Access Protocol \\\hline
VO & Virtual Observatory \\\hline
YAML & Yet Another Markup Language \\\hline
\end{longtable}

% If you want glossary uncomment below -- comment out the two lines above
%\printglossaries





\end{document}
