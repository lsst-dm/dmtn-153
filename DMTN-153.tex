\documentclass[DM,authoryear,toc]{lsstdoc}
% lsstdoc documentation: https://lsst-texmf.lsst.io/lsstdoc.html
\input{meta}

% Package imports go here.

% Local commands go here.

%If you want glossaries
%\input{aglossary.tex}
%\makeglossaries

\title{Schema Management in DM}

% Optional subtitle
% \setDocSubtitle{A subtitle}

\author{%
William O'Mullane
}

\setDocRef{DMTN-153}
\setDocUpstreamLocation{\url{https://github.com/lsst-dm/dmtn-153}}

\date{\vcsDate}

% Optional: name of the document's curator
% \setDocCurator{The Curator of this Document}

\setDocAbstract{%
This note attempts to describe what we mean be schema management, what we would like  and how it is currently implemented within DM. 
}

% Change history defined here.
% Order: oldest first.
% Fields: VERSION, DATE, DESCRIPTION, OWNER NAME.
% See LPM-51 for version number policy.
\setDocChangeRecord{%
  \addtohist{1}{YYYY-MM-DD}{Unreleased.}{William O'Mullane}
}


\begin{document}

% Create the title page.
\maketitle
% Frequently for a technote we do not want a title page  uncomment this to remove the title page and changelog.
% use \mkshorttitle to remove the extra pages

% ADD CONTENT HERE
% You can also use the \input command to include several content files.

\appendix
% Include all the relevant bib files.
% https://lsst-texmf.lsst.io/lsstdoc.html#bibliographies
\section{References} \label{sec:bib}
\renewcommand{\refname}{} % Suppress default Bibliography section
\bibliography{local,lsst,lsst-dm,refs_ads,refs,books}

% Make sure lsst-texmf/bin/generateAcronyms.py is in your path
\section{Acronyms} \label{sec:acronyms}
\addtocounter{table}{-1}
\begin{longtable}{p{0.145\textwidth}p{0.8\textwidth}}\hline
\textbf{Acronym} & \textbf{Description}  \\\hline

 &  \\\hline
ADQL & Astronomical Data Query Language \\\hline
CCB & Change Control Board \\\hline
CI & Continuous Integration \\\hline
CSV & Comma Separated Values \\\hline
DM & Data Management \\\hline
DML & Data Manipulation Model \\\hline
DMTN & DM Technical Note \\\hline
DPDD & Data Product Definition Document \\\hline
FITS & Flexible Image Transport System \\\hline
HSC & Hyper Suprime-Cam \\\hline
IVOA & International Virtual-Observatory Alliance \\\hline
LDM & LSST Data Management (Document Handle) \\\hline
LSE & LSST Systems Engineering (Document Handle) \\\hline
LSP & LSST Science Platform (now Rubin Science Platform) \\\hline
LSST & Legacy Survey of Space and Time (formerly Large Synoptic Survey Telescope) \\\hline
LaTeX & (Leslie) Lamport TeX (document markup language and document preparation system) \\\hline
PPDB & Prompt Products DataBase \\\hline
RFC & Request For Comment \\\hline
SDM & Science Data Model \\\hline
SQL & Structured Query Language \\\hline
TAP & Table Access Protocol \\\hline
VO & Virtual Observatory \\\hline
WCS & World Coordinate System \\\hline
YAML & Yet Another Markup Language \\\hline
\end{longtable}

% If you want glossary uncomment below -- comment out the two lines above
%\printglossaries





\end{document}
