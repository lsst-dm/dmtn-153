
\section{SDM Implementation}

The SDM information must be available to a variety of different tools, each with
slightly different needs. Because of this, it was not sufficient to adopt a format like SQL
\texttt{CREATE TABLE} statements that were only suited to one particular use, and difficult to parse
for all other uses. Instead, the schemas in the \texttt{sdm\_schemas} repository are in a yaml
format defined by the \texttt{Felis} tool. The ease of parsing yaml makes it possible for many
different tools to all share the same source of schema information, minimizing intermediate stages.

The current uses of the Felis-defined yaml files are:
\begin{enumerate}
\item Qserv ingest --- The Felis files are used as inputs to the ingest process.

\item TAP\_SCHEMA creation --- The Felis tool itself is designed to generate SQL statements that
    populate the schema database used by the TAP standard.

\item LDM-153 generation --- The tables in the document generated from the Yaml files.

\item Pipeline data product verification --- Continuous integration tests verify that the pipeline
    outputs comply with the physical schema in \texttt{hsc.yaml}.
\end{enumerate}

Most of these uses depend on the YAML schema files without relying on the Felis tool itself. This is
generally a consequence of the YAML format being easy to parse by other tools.

\subsection{Types of Schemas}

DM maintains multiple SDM specification files, which serve different different organizational
purposes. 

\textbf{``Specification'' schemas} --- LDM-153. This is a schema that is designed to fulfill the
needs of the DPDD's abstract schema, but might not yet be physically realized. This schema is
necessary both for sizing purposes, and as a ``goal'' that the pipelines teams can work to as they
build and evolve the pipeline output files. The result of the construction project should be for the
pipelines to produce data that fully realize this schema.

Because changes to the ``specification'' schema potentially have impacts on multiple areas of DM (e.g.
storage costs or science impacts), it is change-controlled at the DM-CCB level.


\textbf{``Concrete'' schemas} --- e.g. HSC reprocessing schemas. These schemas \textit{are}
physically realized; they are meant to describe data products that currently exist. These schemas
must accurately reflect those data products, regardless of what is specified in the DPDD or LDM-153.
They are not subject to change control since there is no project management impact that can be
caused by any changes. Inaccuracies may cause different dependent services to break, but this is
generally comparable in consequence to any other code breakage.
