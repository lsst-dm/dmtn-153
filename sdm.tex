

\section{SDM requirements}\label{sec:sdm}
The SDM must contain sufficient information for a physical SQL schema definition to be derived from it, given a choice of SQL flavor (e.g., MariaDB, PostgreSQL).

The SDM must contain information that itemizes how it satisfies the DPDD requirements for the content of the data model.  For example, each SDM element that realizes a data item from the DPDD might contain a field that references the appropriate DPDD Identifier.

Each element of the SDM must be described by a unique identifier ("SDM Identifier") that can be used programmatically in applications that consume the SDM YAML definition.  We expect that the "leaf nodes" in the name space of these identifiers will correspond directly to column names in generated database tables; it seems unnecessary to have yet another layer of indirection at this level.  Higher levels in the name space may not correspond exactly to database and/or table names, however; this has yet to be determined precisely.

Software support must be provided for verifying that the SDM provides coverage for all the data items defined in the DPDD.  This should ultimately be subject to verification as part of a CI process.  In order to facilitate the introduction of the SDM language and software into DM, a transitional period should be supported during which a partially complete SDM can be used without triggering constant CI failures.

The design of the SDM and its specification language should address the need to map the physical data model that derives from it to the catalog.schema.table name space of the ADQL 2.0 and TAP standards.  (Bear in mind that the way the term "catalog" is used in this context does not correspond to the intuitive sense of "astronomical catalog".)

In addition to being able to be used to construct a database schema, the SDM specification must also include the information required to provide IVOA-oriented table and column metadata in query responses.  The system must support:

\begin{itemize}
\item      the assignment of UCDs from the UCD1+ standard to all column-like fields in the SDM;
    \item the assignment of IVOA "utypes", where they are useful and taken from either a external standard vocabulary or an LSST-provided vocabulary, to column-like fields as well as, potentially, to tables; and
    \item the definition of "field groups" in the VOTable sense for related data items, such as a quantity and its uncertainty(ies).
\item It should also support:
    mapping of the Science Data Model to externally provided or LSST-defined VO-DML for part or all of its content (at time of writing this appears useful but may be reassessed).
\end{itemize}

The SDM specification must include a data type definition for each column-like field.  This data type is intended to be used to derive several downstream data types involved in the physical instantiation and service of the data:

\begin{itemize}
\item     SQL database types consistent with the variety of actual database software used in LSST, which will include at least MariaDB and PostgreSQL depending on the progress of PPDB development, and should also include the HyperSQL in-memory database used in the Portal Aspect of the LSP (in Firefly);
    \item SQL92-based database types for use in the ADQL context;
    \item VOTable data types;
    \item text-based data formats for use when data model elements are represented as text, e.g., in the VOTable TABLEDATA format or in CSV;
    \item Python data types; and
    \item data types usable in the Parquet and Apache Arrow cross-platform ecosystem, as well as in Apache Avro.
\end{itemize}

It may be necessary for the SDM specification to allow for an override of the "natural" mappings between these target types, so this should be kept in mind in the design, but no specific instance of a need for this has yet been documented.


The SDM specification must contain sufficient information to be able to derive the foreign-key relationships between tables.  It should also allow:
\begin{itemize}
\item the definition of specific columns as required to be indexed (though downstream database implementations may be permitted to add indexes to additional columns for implementation-time performance optimization); and
    \item the definition of the columns and key relationships required to support the Qserv architecture, e.g., the designation of which ra/decl values in a table are the "primary" ones to be used for the spatial partitioning of the table.
\end{itemize}

The SDM specification should be usable to support the mapping of sectors of the Science Data Model to external data models such as ObsCore and CAOM2.  In many cases it may be possible for the SDM itself to include data elements that directly correspond to required elements of these data models; in other cases some conversion may be required.  Both scenarios should be supportable.

The actual mapping to external data models may be part of the SDM specification itself or it may be deferred to additional specification file(s).

The SDM specification "technology" may be used not only to define the LSST data model (for a Data Release, or for the instantiation of the Prompt Processing system and database), but also in narrower contexts to define smaller data models for use in science validation and to support the use of the Science Platform tools during commissioning, for instance.

Concrete use cases that must be supported by the SDM specification and associated software:

\begin{itemize}
\item     generation of executable physical SQL schema, comparable to what is now in the "cat" repository on Github;
\item     definition of the alert data model and its associated Apache Avro .avsc schema;
\item     population of the TAP\_SCHEMA tables required by the TAP 1.1 standard with the information needed to describe the LSST data model, including foreign-key relationships;
\item     generation of the VOTable headers for query results from the LSST TAP service;
\item     population of the tables or, where appropriate, definition of the views mandated by the external standards LSST supports, e.g., the "ivoa.ObsCore" table/view required by the ObsTAP portion of the ObsCore standard, or the standard tables of the CAOM2 data model; and
\item     creation of Parquet files representing the Science Data Model and usable for ingest into the databases.
\end{itemize}

The SDM-to-DPDD mapping information may be useful as part of the documentation of the Science Data Model that LSST exposes to users, to facilitate their understanding of how the SDM corresponds to the DPDD.  To this end, for instance, it may be appropriate to include in the TAP\_SCHEMA tables an additional column (something permitted by the standard) that provides, where appropriate, the DPDD Identifier for a data item.


